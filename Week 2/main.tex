%========Top Matter=========== (fold)
\documentclass[final,expand]{problemset}
% For plot: pfdplots and settings
\usepackage{pgfplots}
\pgfplotsset{width=7cm,
	compat=newest,
	label style={font=\small},
	legend style={font=\small}
}
%========top_matter=========== (end)

\begin{document}
\heading[Caballero]{Sebastián Caballero}{Self Study}{Problem set}{Week 2}

\problem
Let $V$ and $W$ be vector spaces over a field $K$. Let $\mathcal{B} = \{v_1, v_2, \dots,
	v_n\}$ be a basis for $V$ and let $\{w_1, w_2, \dots, w_n\}$ be any
vectors in
$W$. There is a unique linear map \begin{align*}
	\begin{matrix}
		\phi: & V & \to & W \\
	\end{matrix}
\end{align*}
Such that $\phi(v_i) = w_i$ for all $1 \le i \le n$

\solution{
	Since $\mathcal{B}$ is a basis for $V$, for any element $v \in V$ there are $a_1, a_2, \dots, a_n \in K$ such that:
	\begin{align*}
		v &= a_1v_1 + a_2v_2 + \dots + a_nv_n
	\end{align*}
	so if we define $\phi$ such that $\phi(v_i) = w_i$ then for any vector $v$ we would have:
	\begin{align*}
		\phi(v) &= a_1\phi(v_1) + a_2\phi(v_2) + \dots + a_n\phi(a_n)\\
		&= a_1w_1 + a_2w_2 + \dots + a_nw_n
	\end{align*}
}

\problem
Suppose that $V$ is a finite dimensional vector space. Let $\mathcal{B} = \{v_1, v_2, \dots, v_n\}$ be a basis for $V$ then:
\begin{itemize}
	\item Any set of $w_1, w_2, \dots, w_n, w_{n+1}$ vectors is linearly dependent
	\item Any set of $w_1, w_2, \dots, w_{n-1}$ vectors can't generate $V$
\end{itemize}

\solution{
	For this, we are going to use the facts needed for a basis.
	\begin{itemize}
		\item Let $w_1, w_2, \dots, w_n, w_{n+1}$ be vectors in $V$, we can write them in the next way:
		\begin{align*}
			w_1 &= a_{1, 1}v_1 + a_{1, 2}v_2 + \dots + a_{1, n} v_n \\
			w_2 &= a_{2, 1}v_1 + a_{2, 2}v_2 + \dots + a_{2, n} v_n  \\
			\dots & \dots \\
			w_n &= a_{n, 1}v_1 + a_{n, 2}v_2 + \dots + a_{n, n} v_n \\
			w_{n+1} &= a_{n+1, 1}v_1 + a_{n+1, 2}v_2 + \dots + a_{n+1, n} v_n \\
		\end{align*}
		If there is a $w_i$ such that $w_i = 0$ we are done. Suppose then that this is not true, so for each $1 \le i \le n + 1$ exists $j$ such that $a_{i, j} \neq 0$. But since there are $w_{n+1}$ there must be $i_1, i_2$ such that for the same $j$, we have that $a_{i_1, j} \neq 0 \neq a_{i_2, j}$. So, we can express the vector $v_j$ as:
		\begin{align*}
			v_j &= \frac{w_{i_1}}{a_{i_1, j}} - \frac{a_{i_1, 1}v_1 + a_{i_1, 2}v_2 + \dots + a_{i_1, n}v_n}{a_{i_1, j}}\\
			v_j &= \frac{w_{i_2}}{a_{i_2, j}} -\frac{a_{i_2, 1}v_1 + a_{i_2, 2}v_2 + \dots + a_{i_2, n}v_n}{a_{i_2, j}}
		\end{align*}
		And so the set is not linearly independent.

		\item Let $w_1, w_2, \dots, w_{n-1}$ be vectors of $V$. Suppose that indeed we can generate $V$ with them, so in particular, we can write:
		\begin{align*}
			v_1 &= a_{1, 1}w_1 + a_{1, 2}w_2 + \dots + a_{1, {n-1}} w_{n-1} \\
			v_2 &= a_{2, 1}w_1 + a_{2, 2}w_2 + \dots + a_{2, {n-1}} w_{n-1}  \\
			\dots & \dots \\
			v_n &= a_{n, 1}w_1 + a_{n, 2}w_2 + \dots + a_{n, {n-1}} w_{n-1} \\
		\end{align*}
		And since none of them is zero, we can be fure that for each $1 \le i \le n$ exists $j$ such that $a_{i, j} \neq 0$. But since there are $n$ vectors in $\mathcal{B}$ and just $n-1$ vectors $w_i$, there must be $i_1, i_2$ such that for the same $j$, we have that $a_{i_1, j} \neq 0 \neq a_{i_2, j}$. So, we can express the vector $v_j$ as:
		\begin{align*}
			w_j &= \frac{v_{i_1}}{a_{i_1, j}} - \frac{a_{i_1, 1}w_1 + a_{i_1, 2}w_2 + \dots + a_{i_1, n}w_n}{a_{i_1, j}}\\
			w_j &= \frac{v_{i_2}}{a_{i_2, j}} -\frac{a_{i_2, 1}w_1 + a_{i_2, 2}w_2 + \dots + a_{i_2, n}w_n}{a_{i_2, j}}
		\end{align*}
		But then this let us generate two different linear combinations within $\mathcal{B}$ that give us the same result, contradicting the linear independency of $\mathcal{B}$.
	\end{itemize}
}

\problem Let $V$ be a finite vector space. If $A = \{v_1, v_2, \dots, v_n\}$ generates $V$ then some subset of $A$ is a basis for $V$.

\solution{
    For that, let declare the next set:
    \begin{align*}
        S &= \{W \in \mathcal{P}(A) | W \text{ is linearly independent}\}
    \end{align*}
    We can assure that at least there is a maximal element $\{v_1, v_2, \dots, v_m\}$ in $S$ since we can assure the existence of $\{v_1\}$ and at most it can be $A$. Suppose then that it is not $A$, so $m < n$, and we can assure that any set $\{v_1, \dots, v_m, v_i\}$ is linearly dependent, with $m < i \le n$. Therefore we have:
	\begin{align*}
		a_1v_1 + \dots + a_nv_n + a_iv_i &= 0
	\end{align*}
	has more than the trivial solution, so we can suppose that 
}

\problem
Let $A = \{v_1, v_2, \dots, v_n\}$ be a subset of a vector space $V$. Prove that $A$ is linearly independent if and only if the equation $a_1v_1 + a_2v_2 + \dots + a_nv_n = 0$ has the trivial solution.

\solution{
	We prove a double implication:
	\begin{itemize}
		\item[$\Rightarrow)$] If $A$ is linearly independent then by definition the equation $a_1v_1 + a_2v_2 + \dots + a_nv_n = 0$ has only one solution, the trivial one.
		\item[$\Leftarrow)$] Suppose that $A$ is not linearly independent, so that there are two combinations of scalars $a_1, a_2, \dots, a_n$ and $b_1, b_2, \dots, b_n$ such that for a $v$ in $V$:
		\begin{align*}
			a_1v_1 + a_2v_2 + \dots + a_nv_n &= v\\
			b_1v_1 + b_2v_2 + \dots + b_nv_n &= v
		\end{align*}
		And if we use the transitivity we have:
		\begin{align*}
			a_1v_1 + a_2v_2 + \dots + a_nv_n &= b_1v_1 + b_2v_2 + \dots + b_nv_n\\
			(a_1 - b_1)v_1 + (a_2 - b_2)v_2 + \dots + (a_n - b_n)v_n &= 0
		\end{align*} 
		But note that $a_1 \neq b_1$, $a_2 \neq b_2$ and so on, so $a_1 - b_1 \neq 0$, $a_2 - b_2 \neq 0$ and so on, so the equation has another solution apart to the trivial one.
	\end{itemize}
}

\problem
Let $W$ be a subspace of a finite dimensional vector space $V$. Any basis for $W$ can be extended to a basis for $V$.

\problem
Prove the Rank theorem

\problem
Determine whether or not $\{(1, 1, 0), (2, 0, -1), (-3, 1, 1)\}$ is basis for $\mathbb{R}^3$

\problem
Let $\phi: V \to W$ be linear. Suppose that $v_1, \dots, v_n \in V$ are such that $\phi(v_1), \dots, \phi(v_n)$ are linearly independent in $W$. Show that $v_1, \dots, v_n$ are linearly independent.

\solution{
	For that, since $\phi(v_1), \dots, \phi(v_n)$ are linearly independent, we can assure that the equation:
	\begin{align*}
		a_1\phi(v_1) + a_2\phi(v_2) + \dots + a_n\phi(v_n) &= 0
	\end{align*}
	has only the trivial solution. Suppose that the equation:
	\begin{align*}
		b_1v_1 + b_2v_2 + \dots + b_nv_n &= 0
	\end{align*}
	has a solution that is not trivial. That this, we can assure that at least $b_1$ is not $0$. And if we apply to both sides the linear map $\phi$ we get:
	\begin{align*}
		\phi(b_1v_1 + b_2v_2 + \dots + b_nv_n) &= \phi(0)\\
		\phi(b_1v_1) + \phi(b_2v_2) + \dots + \phi(b_nv_n) &= 0\\
		b_1\phi(v_1) + b_2\phi(v_2) + \dots + b_n\phi(v_n) &= 0
	\end{align*}
	But this is a contradiction since this equation can only have the trivial solution. So we can conclude that $v_1, \dots v_n$.
}

\problem
Let $V$ be a vector space over a field $k$ and let $U, W$ be finite dimensional subspaces of $V$. Prove that $U + W$ and $U \cap V$ are finite dimensional subspaces of $V$ and:
\begin{align*}
	\dim(U + W) + \dim(U \cap W) &= \dim U + \dim W
\end{align*}

\problem
Show that the set of real numbers $\mathbb{R}$ is a vector space over the rational numbers $\mathbb{Q}$. Show that this is not a finite-dimensional vector space over $\mathbb{Q}$.

\problem
If $\{v_1, \dots, v_n\}$ is a basis for $V$ and $\{w_1, \dots, w_m\}$ is a basis for $W$ then:
\begin{align*}
	\{(v_1, 0), \dots, (v_n, 0), (0, w_1), \dots, (0, w_n)\}
\end{align*}
is a basis for $V \oplus W$

\solution{
	We need to prove two things:
	\begin{itemize}
		\item First, to prove that this set is linearly independent, we need to show that the homogeneous equation has only the trivial solution. So we have:
		\begin{align*}
			a_1(v_1, 0) + a_2(v_2, 0) + \dots + a_n(v_n, 0) + b_1(0, w_1) + b_2(0, w_2) + \dots + b_n(0, w_n) &= (0, 0)\\
			(a_1v_1, 0) + (a_2v_2, 0) + \dots + (a_nv_n, 0) + (0, b_1w_1) + (0, b_2w_2) + \dots + (0, b_nw_n) &= (0, 0)\\
			(a_1v_1 + a_2v_2 + \dots  +a_nv_n, b_1w_1 + b_2w_2 + \dots + b_nw_n) &= (0, 0)
		\end{align*}
		And this means that:
		\begin{align*}
			a_1v_1 + a_2v_2 + \dots  +a_nv_n &= 0\\
			b_1w_1 + b_2w_2 + \dots + b_nw_n &= 0
		\end{align*}
		And since those vectors are basis for each vector space $a_1 = a_2 = \dots = a_n = b_1 = b_2 = \dots = b_n$.

		\item For an element $(v, w) \in V \oplus W$, we know that $v$ can be expressed as a linear combination $a_1v_1 + a_2v_2 + \dots + a_nv_n = v$, and also $w$ can be expressed as $b_1w_1 + b_2w_2 + \dots + b_nv_n = w$, so the combination of the vectors in our set will rise:
		\begin{align*}
			a_1(v_1, 0) + a_2(v_2, 0) + \dots + a_n(v_n, 0) + b_1(0, w_1) + b_2(0, w_2) + \dots + b_n(0, w_n) = (v, w)
		\end{align*}
	\end{itemize}
}

\problem
Let $V$ be a vector space over $K$ and let $W$ a subspace of $V$. With the  operations induced over $V / W$, this becomes a vector space.

\problem
Generalize the notions of external and internal direct sums to three or more summands.

\solution{
	For a set $V_1, V_2, \dots, V_n$ of vector spaces defined over the same field $K$, we define the external direct sum as:
	\begin{align*}
		\bigoplus_{i = 1}^n V_i &= \prod_{i = 1}^n V_I
	\end{align*}
	With the operations defined as:
	\begin{align*}
		(v_1, v_2, \dots, v_n) + (w_1, w_2, \dots w_n) &= (v_1 + w_1, v_2 + w_2, \dots, v_n + w_n) \\c(v_1, v_2, \dots, v_n) &= (cv_1, cv_2, \dots, cv_n)
	\end{align*} 
	which also produce a vector space. And let $V$ be a vector space, $\mathcal{V} = \{V_1, V_2, \dots, V_n\}$ a collection a subspaces of $V$. We define that $V$ is the internal sum of the elements of $\mathcal{V}$ if and only if there is a function $\eta: \bigoplus\limits_{i = 1}^n V_i \to V$ such that:
	\begin{itemize}
		\item $\eta(v_1, v_2, \dots, v_n) = v_1+v_2+\dots + v_n$
		\item $\eta$ is monic
		\item $\eta$ is epic
	\end{itemize}
}

\problem
Let $W$ be a subspace of the finite-dimensional vector space $V$. Show that there is a subspace $U$ of $V$ such that $V \cong U \oplus W$.

\solution{
	For this, define $U$ as follows:
	\begin{align*}
		U &:= V \setminus W \cup \{0\}
	\end{align*}
	First, we need to prove that this is a subspace of $V$:
	\begin{quote}
		Note that for any $v \in U$ different from $0$ and any $c \in K$, if $cv \in W$ then $c^{-1}cv = v \in W$ which contradicts the definition of $U$. If $u, w \in U$ are not both $0$, and if $u + w \in W$ then that means that $u, w \in W$ since $W$ is closed over the operations, which again, contradicts the definition for $U$, so $u + w \in U$. 
	\end{quote}

	Now, we want to prove that this is an internal sum of $V$, so we have:
	\begin{itemize}
		\item If $w \in W$ and $u \in U$ are such that $w + u = 0$, then we would have $w = -u$, which means that $w \in U$ and also that $u = -w \in V$, which means that since its only common element is $0$, $u = w = 0$.
		\item For any element $v \in V$, there are two alternatives. If $v \in W$ then we can express $v$ as $v + 0$ and $0 \in U$. If $v \not\in W$ then $v \in U$ by definition and so $v = 0 + v$ with $0 \in W$. 
	\end{itemize}	

	And so we conclude that $U \oplus W$ is an internal sum of $V$.
}

\problem
Prove that every vector space has a basis.

\problem
Prove that the set of all infinite sequences of $0's$ and $1's$ with component-wise addition and scalar multiplication modulo 2.

\problem
Can $\mathbb{C}$ be isomorphic to a subspace of $\mathbb{R}$?

\problem
Prove the binomial and the multinomial theorem for rings


\end{document}